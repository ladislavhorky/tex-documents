% -*-coding: utf-8 -*-
\documentclass[12pt, a4paper]{report}

% JAZYK, FONTY, LAYOUT
\usepackage[czech]{babel}
\usepackage[utf8]{inputenc}
\usepackage[T1,IL2]{fontenc}
\usepackage[margin=1in]{geometry}

% MATEMATICKÉ DEFINICE
% POUŽÍVEJTE PROSTŘEDÍ theorem, lemma, definition, remark, example, NEBO SI DEFINUJTE NOVÉ
\usepackage{amsmath,amsfonts,amsthm,mathtools,upgreek}
\def\proofname{{\scshape Důkaz.}}

\begin{document}

Ladislav Horký \hfill 16.2.2012
\section{Lehký příklad}
Spočtěte (absolutně přesně a bez hrubé síly) mocninu $A^{1000}$ k matici
\[ A =
\begin{pmatrix}
\quad3/2 & 1/2  \\
-1/2 & 1/2
\end{pmatrix} =
\frac{1}{2}
\begin{pmatrix}
\quad3 & 1  \\
-1 & 1
\end{pmatrix}.
\]

\vspace{1cm}
Zkusme několik kroků:
\[ A^3 =
\begin{pmatrix} \quad3/2 & 1/2  \\ -1/2 & 1/2 \end{pmatrix}
\begin{pmatrix} \quad3/2 & 1/2  \\ -1/2 & 1/2 \end{pmatrix}
\begin{pmatrix} \quad3/2 & 1/2  \\ -1/2 & 1/2 \end{pmatrix} =
\]
\[
 = \begin{pmatrix} \quad2 & 1  \\ -1 & 0 \end{pmatrix}
\begin{pmatrix} \quad3/2 & 1/2  \\ -1/2 & 1/2 \end{pmatrix} =
\begin{pmatrix} \quad5/2 & \quad3/2  \\ -3/2 & -1/2 \end{pmatrix}.
\]
Po několika krocích máme hypotézu:
\[ A^{k+1} = \frac{1}{2}
\begin{pmatrix} \quad3+k & 1+k  \\ -1-k & 1-k \end{pmatrix}.
\]
\begin{proof}
Pro $k=1,2$ hypotéza evidentně platí. Pokračujme v indukci $k \rightarrow k+1$:
\[ A^{k+1} = A^k A = \frac{1}{2}
\begin{pmatrix} \quad3+(k-1) & \;1+(k-1)  \\ -1-(k-1) & \;1-(k-1) \end{pmatrix}
\begin{pmatrix} \quad3/2 & 1/2  \\ -1/2 & 1/2 \end{pmatrix} =
\]
\[
 = \frac{1}{4} \begin{pmatrix} \quad6+2k & \quad2+2k  \\ -2-2k & \quad2-2k \end{pmatrix} =
 \frac{1}{2}
\begin{pmatrix} \quad3+k & 1+k  \\ -1-k & 1-k \end{pmatrix} = A^{k+1}.
\]
\end{proof}

Z toho plyne pro náš případ

\[ A^{1000} = A^{999+1} = \frac{1}{2}
\begin{pmatrix} \quad1002 & \quad1000  \\ -1000 & -998 \end{pmatrix} =
\begin{pmatrix} \quad501 & \quad500  \\ -500 & -499 \end{pmatrix}.
\]


\end{document} 