% -*-coding: utf-8 -*-
\documentclass[11pt,a4paper]{report}

\newif\ifrelease % new boolean variable release. True = include some fancy content
\releasefalse % and set it

\usepackage[czech]{babel}
\usepackage[utf8]{inputenc}
\usepackage[IL2]{fontenc}

\usepackage{amsmath}
\usepackage{amsthm}
\usepackage{amstext}
\usepackage{amsfonts}
\usepackage{epsfig}
\usepackage{graphicx}
\usepackage{color}
\usepackage{booktabs}
%\usepackage[justification=centering]{caption}
\ifrelease
	\usepackage{pdfpages}
\fi
\usepackage[pagebackref=true]{hyperref} % tento balicek by mel byt na konci baliku!

\hypersetup{
	pdfauthor={Ladislav Horký},
	pdftitle={Použití fuzzy logiky ve zpracování obrazu na GPU}
}

%% Nastavení zrcadla sazby
\usepackage{calc}
\setlength{\textheight}{9in}
\setlength{\textwidth}{6in}
\setlength\oddsidemargin{(\paperwidth-\textwidth)/2 - 1in}
\setlength\evensidemargin{(\paperwidth-\textwidth)/2 - 1in}
\setlength\topmargin{(\paperheight-\textheight-\headheight-\headsep-\footskip)/2 - 1in}

%\parindent=0pt % odsazení 1. řádku odstavce
%\parskip=4pt   % mezera mezi odstavci

\ifrelease
	\hypersetup{pdfborder={0 0 0}} % no borders around links
\else
	\hypersetup{colorlinks=true} % colour links instead of borders
\fi
% =========================================================
% Ams definice

\theoremstyle{plain}
\newtheorem{define}{Definice}
\newtheorem{theo}{Věta}


% Vlastní příkazy:==========================================
\newif\ifshownotes % zobraz poznámky level 1
\shownotesfalse
\newif\ifshownotesadd % zobraz poznámky level 2
\shownotesaddtrue

\newcommand{\LAs}{\L A_{sqrt}}
\newcommand{\rl}{\textup{(}}
\newcommand{\rr}{\textup{)\,}}
\ifshownotes
    \newcommand{\note}[1]{{\color{green}{\emph{#1}}}}
\else
    \newcommand{\note}[1]{}
\fi
\ifshownotesadd
    \newcommand{\notea}[1]{{\color{green}{\emph{#1}}}}
\else
    \newcommand{\notea}[1]{}
\fi
\newcommand{\LL}{\mathbf{L}}
\newcommand{\sqr}{\mathrm{sqrt}}
\newcommand{\LAsq}{$\mathrm{\L A_{sqrt}}$}
\newcommand{\beq}{\begin{equation}}
\newcommand{\eeq}{\end{equation}}
\newcommand{\xx}{\mathbf{x}}
\newcommand{\yy}{\mathbf{y}}
\newcommand{\f}{\mathrm{f}}
\newcommand{\g}{\mathrm{g}}
\newcommand{\PP}{\mathcal{P}}
\newcommand{\QQ}{\mathcal{Q}}
\newcommand{\NN}{\mathcal{N}_{R}(x_{i,j,k})}
\newcommand{\MM}{\mathcal{M}_R}
\newcommand{\Lw}{{\mathrm{L^w}_{i,j,k}}}
\newcommand{\Nb}{{\mathrm{N}_{i,j,k}}}
\newcommand{\WL}{{\mathrm{WL}_{i,j,k}}}
\newcommand{\EE}{\mathrm{E}}
\newcommand{\DD}{\mathrm{D}}
\newcommand{\OO}{\mathrm{O}}
\newcommand{\CC}{\mathrm{C}}
\newcommand{\MED}{\mathrm{MED}}
\newcommand{\BES}{\mathrm{BES}}
\newcommand{\kk}{\textit{k}}
\newcommand{\et}{\;\,\mathrm{et}\;\,}

% Definice makra pro české uvozovky:
\def\bq{\mbox{\kern.1ex\protect\raisebox{-1.3ex}[0pt][0pt]{''}\kern-.1ex}}
\def\eq{\mbox{\kern-.1ex``\kern.1ex}}
\def\ifundefined#1{\expandafter\ifx\csname#1\endcsname\relax }%
\ifundefined{uv}%
        \gdef\uv#1{\bq #1\eq}
\fi

% Dělení slov:=============================================
%\hyphenation{t-norma}  JAK KRUCI ABY TO NEDĚLILO?

% =========================================================
\begin{document}

\ifrelease
	% -*-coding: utf-8 -*-
\newcommand{\cvut}{České Vysoké Učení Technické v~Praze}
\newcommand{\fjfi}{Fakulta Jaderná a Fyzikálně Inženýrská}
\newcommand{\km}{Katedra matematiky}
\newcommand{\obor}{Inženýrská Informatika}
\newcommand{\zamereni}{Softwarové Inženýrství a Matematická Informatika}

\newcommand{\nazevcz}{Použití fuzzy logiky ve zpracování obrazu na GPU}
\newcommand{\nazeven}{---}
\newcommand{\autor}{Ladislav Horký}
\newcommand{\rok}{2011}
\newcommand{\vedouci}{Ing. Tomáš Oberhuber Ph.D.}

\newcommand{\pracovisteVed}{Katedra matematiky \\
    České Vysoké Učení Technické v~Praze}
\newcommand{\konzultant}{doc. Ing. Jaromír Kukal Ph.D.}
\newcommand{\pracovisteKonz}{Katedra softwarového inženýrství \\
    České Vysoké Učení Technické v~Praze}

\newcommand{\klicova}{TODO, klíčová slova, max. 5}
\newcommand{\keyword}{TODO, key, words}
\newcommand{\abstrCZ}{TODO Abstrakt práce (cca 7 vět, min. 80 slov)}
\newcommand{\abstrEN}{TODO English abstract}

%%% zde zacina kresleni dokumentu

% titulní strana
\thispagestyle{empty}

\begin{center}
	{\Large  \bf  \cvut\\[2mm] \fjfi }
	\vspace{10mm}

	\begin{tabular}{c}
	{\bf \km}\\
	{\bf Obor: \obor}\\
	{\bf Zaměření: \zamereni}
	\end{tabular}

	\vspace{10mm} \epsfysize=20mm  \epsffile{cvut-logo-bw-600} \vspace{15mm}

	{\LARGE
	\textbf{\nazevcz}
	\par}

	\vspace{5mm}

	{\LARGE
	\textbf{\nazeven}
	\par}

	\vspace{30mm}
	{\Large BAKALÁŘSKÁ PRÁCE}

\end{center}

\vfill
{\large
\begin{tabular}{rl}
Vypracoval: & \autor\\
Vedoucí práce: & \vedouci\\
Rok: & \rok
\end{tabular}
}

% zadání bakalářské práve
\newpage
\thispagestyle{empty} Před svázáním místo téhle strany \fbox{vložíte zedání práve} s podpisem
děkana (bude to jediný oboustranný list ve Vaší práci) !!!!

% prohlášení
\newpage
\thispagestyle{empty}
~
\vfill


{\bf Prohlášení}

\vspace{0.5cm}
Prohlašuji, že jsem svou bakaářskou práci vypracoval samostatně a použil jsem pouze podklady
(literaturu, projekty, SW atd.) uvedené v přiloženém seznamu.

\vspace{5mm}V Praze dne ....................\hfill
    \begin{tabular}{c}
    ........................................\\
    \autor
    \end{tabular}

% poděkování
\newpage
\thispagestyle{empty}
~
\vfill

{\bf Poděkování}

\vspace{5mm}
Děkuji ... ...

\begin{flushright}
\autor
\end{flushright}

% strana s abstraktem
\newpage
\thispagestyle{empty}

\newbox\odstavecbox
\newlength\vyskaodstavce
\newcommand\odstavec[2]{%
    \setbox\odstavecbox=\hbox{%
         \parbox[t]{#1}{#2\vrule width 0pt depth 4pt}}%
    \global\vyskaodstavce=\dp\odstavecbox
    \box\odstavecbox}
\newcommand{\delka}{120mm}

\begin{tabular}{ll}
  {\em Název práce:} & ~ \\
  \multicolumn{2}{l}{\odstavec{\textwidth}{\bf \nazevcz}} \\[5mm]
  {\em Autor:} & \autor \\[5mm]
  {\em Obor:} & \obor \\
  {\em Druh práce:} & Bakalářská práce \\[5mm]
  {\em Vedoucí práce:} & \odstavec{\delka}{\vedouci \\ \pracovisteVed} \\[5mm]
  {\em Konzultant:} & \odstavec{\delka}{\konzultant \\ \pracovisteKonz} \\[5mm]
  \multicolumn{2}{l}{\odstavec{\textwidth}{{\em Abstrakt:} ~ \abstrCZ \\ }} \\[5mm]
  {\em Klíčová slova:} & \odstavec{\delka}{\klicova} \\[20mm]

  {\em Title:} & ~\\
  \multicolumn{2}{l}{\odstavec{\textwidth}{\bf \nazeven}}\\[5mm]
  {\em Author:} & \autor \\[5mm]
  \multicolumn{2}{l}{\odstavec{\textwidth}{{\em Abstract:} ~ \abstrEN \\ }} \\[5mm]
  {\em Key words:} & \odstavec{\delka}{\keyword}
\end{tabular}
 % úvodní strany
\fi

    \tableofcontents

    \addcontentsline{toc}{chapter}{Úvod}
    \chapter*{Úvod}
        % -*-coding: utf-8 -*-


Heuristické algoritmy jsou silným a v praxi mnohdy i jedniným nástrojem k řešení optimalizačních problémů s vysokou, často exponenciální složitostí v případech, kdy není známo analytické řešení. Když rozměr problému roste, náročnost algoritmů pak rychle stoupá nad únosné meze (jako např. při učení neuronových sítí) a my jsme nuceni buď vymyslet lepší algoritmus, zjednodušit problém, nebo dát stávajícímu algoritmu více výpočetních prostředků. Zatímco první dvě možnosti jsou netriviální, nebo nežádoucí, třetí je v praxi dobře realizovatelná, i když složitost zlepší jen o konstantu. Zajímavou třídou optimalizačních úloh \note{proč zajímavou? je nutné to psát?} jsou úlohy z podstaty celočíselné (Knapsack, Sudoku), kde nemůžeme použít algoritmy vyžadující spojitost účelové funkce. Právě touto třídou se budeme v práci zabývat především.


Jedním ze způsobů jak výpočty urychlit je paralelizace na GPU v prostředí NVIDIA CUDA. Jako motivace nám může sloužit několik konkrétních heuristických algoritmů (\note{citace algoritmů}), které byly s dobrými výsledky na GPU již implementovány. Náš cíl je však obecnější: chceme vytvořit obecný formalismus, v kterém budeme moci elegantním způsobem popsat (dekomponovat) co nejširší škálu optimalizačních algoritmů. K tomu je třeba nejprve známé algoritmy analyzovat a hledat jejich společné rysy podobně, jak je to popsáno v \cite{GO ebook}. Důležitou stránkou formalismu je i jeho ohled na implementaci, neboť paralelizace na GPU je značně hardwarově i softwarově specifická a pro optimální výsledek musíme dodržet řadu omezení.

%chceme vytvořit dobře paralelizovatelný model, v němž bude možné elegantním způsobem formulovat (dekomponovat) co nejširší škálu heuristických algoritmů \note{spíš model, v němž je bude možné snadno \emph{implementovat} .. po mírné reformulaci, která neovlivní funkci}. K tomu je třeba nejprve analyzovat známé algoritmy, hledat jejich společné rysy a nalézt pro ně společný formalismus - podobně jak je to popsáno v \cite{GO ebook}. Druhou důležitou stránkou je ohled modelu na implementaci, neboť paralelizace na GPU je značně hardwarově i softwarově specifická a pro optimální výsledek musíme dodržet řadu omezení.


V dalších částech se tedy budeme věnovat analýze několika známých heuristických algoritmů, specifikům paralelizace na GPU a formulaci požadavků, které ve světle předchozích dvou témat na náš formalismus a programovací model klademe. Poté popíšeme samotný model, jeho praktickou implementaci a testování. 

    \chapter{Fuzzy logika}
        % Lukasiewiczova algebra s odmocniou
            % Konstrukce matematického modelu
                % Svaz
                % Lukasiewiczova algebra
                % Lukasiewiczova algebra s odmocninou
            % Fuzzy-logická funkce
            % Používané operátory
        % Citlivost
            % Přesnost celočíselné aritmetiky
        % Reprezentace obrazových dat
            % Maska
        % Filtry
            % Typologie filtrů
                % Morfologické
                % Statistické
            % Citlivost filtrů
            % Porovnání s lineárními filtry

        % -*-coding: utf-8 -*-

    % Lukasiewiczova algebra s odmocniou
            % Konstrukce matematického modelu
                % Svaz
                % Lukasiewiczova algebra
                % Lukasiewiczova algebra s odmocninou
            % Fuzzy-logická funkce
            % Používané operátory
        % Citlivost
            % Přesnost celočíselné aritmetiky
        % Reprezentace obrazových dat
            % Maska
        % Filtry
            % Typologie filtrů
                % Morfologické
                % Statistické
            % Citlivost filtrů
            % Porovnání s lineárními filtry
%==============================================================


        \note{fuzzy -- dvouhodnotová logika je málo - příliš velká ztráta informace, struktura logiky je vhodná pro analýzu vlastností, umožňuje redukovat
        magické konstanty a zároveň je dostačující pro většinu praktických úloh}


\section{\L ukasiewiczova algebra s odmocninou}
    % bez odmocniny bude množina filtrů omezená (uzavření-otevření nic nedělá) -- vyhnout se přímo trvzení o konečnosti, nekonečnosti
                % jen svaz s neg, V, A má 2^(2^n) operací
                %%%% z residua a negace
        % obohatit systém tak, aby se dala snížit citlivost a dělat průměry -- prostě co ejmenší množina operací, se kterými už uděláme, co potřebujeme
        % odmocnina jde vždy propasovat dolů, jde jí předejít vydělením vstupů mocninou 2ky (2a, 4a udělám pomocí krát)

        % porovnání s lineárními filtry: (Gauss, atd)
            % lineární jsou lepší na gaussovský šum
            % fuzzy umožňuje ubrat předpoklady o vstupu (druh šumu, SMĚS ŠUMŮ -- KOSMICKÉ ZÁŘENÍ VE SPECT...)
    % obecně robustní filtry (dá se udělat fuzzy k-tý prvek)
    % při algoritmickém generování sítí, instantním prohlížení a filtrace
    %=============================================================================================================================

    V této části zkonstruujeme matematický model, který bude snadno aplikovatelný na obrazová data ve formě, jak s nimi pracuje počítač a zároveň nám poskytne vhodný základ pro konstrukci filtrů. Budeme postupovat podobně jako v \cite{MajerovaPhD} a ukážeme, že \L ukasiewiczova algebra s odmocninou je vhodným kandidátem. Na model přitom klademe následující poždavky:
    \begin{itemize}
      \item maximální jednoduchost matematických operací
      \item snadná realizovatelnost v celočíselném oboru
      \item omezená citlivost
      \item neomezená množina filtrů, které je možno zkonstruovat
      \item možnost konstrukce robustních\footnote{tzn. do výstupu filtru se (příliš) nepromítnou okrajové extrémní hodnoty ze vstupu} nízkofrekvenčních filtrů
      \item možnost konstrukce filtrů na bázi aritmetického průměru
      \item možnost konstrukce vysokofrekvenčních filtrů (detektory hran, segmentace)
    \end{itemize}
    Z poždavku jednoduchosti je jasné, že budeme postupovat od základních matematických struktur a ty postupně obohacovat o další operace.

    \subsection{Matematický model}

    Obraz je v počítači reprezentován jako sada diskrétních hodnot omezených v nějakém intervalu (0.0 až 1.0, 0 až 255...). Struktura musí být tedy uzavřená vůči vlastním operacím, musí mít jako nosič omezenou množinu a neměla by jí vadit ani diskretizace této množiny.

    \begin{define}\label{svaz}
    Algebru $S = (M,\wedge,\vee)$ se dvěma binárními operacemi takovými, že platí:
    \begin{align}
    a \wedge b &= b \wedge a, & a \vee b &= b \vee a &&\text{\rl komutativní zákon\rr} \\
    (a \wedge b) \wedge c &= a \wedge (b \wedge c), & (a \vee b) \vee c &= a \vee (b \vee c) &&\text{\rl asociativní zákon\rr} \\
    a \wedge (b \vee a) &= a,& a \vee (b \wedge a) &= a &&\text{\rl zákon absorpce\rr}
    \end{align}
    nazýváme \textbf{svazem}. Operace $\wedge$ \rl\textbf{průsek}\rr a $\vee$ \rl\textbf{spojení}\rr dodefinujeme v souladu s teorií:
    \begin{align}
    a \wedge b &= \min(a,b), \\
    a \vee b &= \max(a,b).
    \end{align}
    \end{define}

    Svaz nám bez dalšího obohacení umožňuje konstruovat základní morfologické filtry jako eroze a dilatace\footnote{tyto a všechny další filtry budou vysvětleny v sekci~\ref{Filtry}, zde slouží pouze pro ilustraci} a je nad ním možno zkonstruovat libovolnou pořadovou statistiku (\kk-tý prvek) -- základ mnoha robustních filtrů ostraňujících šum:

    \begin{define}\label{def k-prvek}
      Buď $n \in \mathbb{N}$, $x_i \in \mathbb{R}, \,\forall i \in \widehat n$. Buď $(x_1,x_2,...,x_n)$ seřazený soubor hodnot tak, že $x_1 \geq x_2 \geq ... \geq x_n$. Pak $x_k$ nazýváme \textbf{k-tou pořadovou statistikou} (k-tým prvkem) tohoto souboru.
    \end{define}

    \begin{theo}\label{theo k-tý prvek}
      Nechť $x_1,x_2,...,x_n \in M$, dále buď $l = n-k+1$, $C^{l}_n$ oindexovaná množina všech l-tic z $\widehat n$, $c^i \in C^{l}_n$ a $c^i_j$ je j-tý prvek i-té l-tice. Pak
      \[
        a_k = \bigwedge_{i = 1}^{{l}\choose{n}}\left( \bigvee_{j = 1}^{l} x_{c_{j}^i} \right)
      \]
      je k-tá pořadová statistika souboru $(x_1,x_2,...,x_n)$.
    \end{theo}
    \begin{proof}
      Vnitřní operace vybere postupně ze všech \textit{l}-tic maximum, vnější operace pak vybere nejmenší z těchto maxim. Nejmenšího maxima se dosáhne právě tehdy, chybí-li v \textit{l}-tici největších $k-1$ prvků a tudíž tímto maximem bude tedy právě \kk-tý prvek.
    \end{proof}

    Bohužel už medián ze sudého počtu prvků je nerealizovatelný (chybí nám aritmetický průměr), stejně jako jakýkoliv vysokofrekvenční filtr\footnote{například detektor hran} (chybí nám rozdíl). Z povahy svazových operací je navíc zřejmé, že získáme pouze hodnotu, která už se objevila na vstupu.

    Abychom získali možnost odčítat, obohatíme svaz o dvě binární operace, z nichž se dá operace rozdílu odvodit (viz \ref{operátory}):

    \begin{define}\label{def residuovaný svaz}
    Buď $(M,\wedge,\vee)$ je svaz s nejmenším prvkem 0 a největším 1, $\otimes$ \rl \textbf{součin}, nebo \textbf{t-norma}\rr je binární, asociativní a komutativní operace a $\rightarrow$ \rl \textbf{residuum}\rr je binární operace. Navíc platí:
    \begin{align}
    &x \otimes 1 = x  &&(\forall x \in M)\\
    &x \otimes y \leq z \Leftrightarrow y \rightarrow z \geq x &&(\forall x,y,z \in M)\label{Galoisova koresp}
    \end{align}
    Pak strukturu $(M,\wedge,\vee,\otimes,\rightarrow,0,1)$ nazýváme \textbf{residuovaný svaz}.
    \end{define}

    Galoisova korespondence \eqref{Galoisova koresp} přiřazuje každému součinu právě jedno residuum \cite{MajerovaPhD}, k plnému popisu operací tedy stačí definovat pouze součin. Za ten zvolíme \L ukasiewiczovu t-normu:

    \begin{define}\label{LA}
    Residuovaný svaz $(\LL,\wedge,\vee,\otimes,\rightarrow,0,1)$ s operacemi $\otimes$ a $\rightarrow$ definovanými v souladu s \eqref{Galoisova koresp} jako
    \begin{align}
    x \otimes y &= \max(x+y-1,0) \\
    x \rightarrow y &= \min(1,1-x+y)
    \end{align}
    a nosičem $\LL = [0,1]$ \rl 0,1 zde opět bereme jako obecný nejmenší a největší prvek, interval později ztotožníme s rozsahem intenzit voxelů v obraze\rr \note{co \ref{def obraz}? je to takhle elegantní?} nazýváme \textbf{\L ukasiewiczova algebra} \textup{(\L A)}.
    \end{define}

    \begin{theo}
      \textup{(\L A)} splňuje podmínky residuovaného svazu.
    \end{theo}
    \begin{proof}
    První rovnost platí triviálně:
      \[
      x \otimes 1 = \max(x+1-1,0) = \max(x,0) = x
      \]
    V druhé ekvivalenci využijeme toho, že pro $\forall x,y \in \LL$ platí $\max(x,y) = -\min(-x,-y)$ a $x,y,z \geq 0$:
    \begin{align*}
      x \otimes y \leq z &\Leftrightarrow \max(x+y-1,0) \leq z \Leftrightarrow \max(x+y-1-z,-z) \leq 0 \Leftrightarrow\\
      &\Leftrightarrow \min(1+z-x-y,z) \geq 0 \Leftrightarrow \min(1+z-y,z+x) \geq x \Leftrightarrow\\
      &\Leftrightarrow 1+z-y \geq x \et z+x \geq x \Leftrightarrow\\
    \intertext{Poslední nerovnost platí vždy. Můžeme ji tedy vynechat, nebo pro naše účely vyměnit za jinou tautologii $1 \geq x$:}
      &\Leftrightarrow 1+z-y \geq x \et 1 \geq x \Leftrightarrow \min(1+z-y,1) \geq x \Leftrightarrow\\
      &\Leftrightarrow y \rightarrow z \geq x
    \end{align*}
    \end{proof}

    \note{0,1 zde představuje jen největší a nejmenší prvek -- je nutné to zmínit?}
    \note{Proč potřebujeme MV-algebru? Gougen, Godel ani Yager nejsou --- říká,že neg neg x je x, což je slušné chování}

    Existují i jiné t-normy (\cite{MajerovaPhD},\cite{Bělíček}), ty ovšem nemají vlastnosti mnohahodnotové logiky \cite{MajerovaPhD}, nebo jsou příliš výpočetně náročné. Poslední obohacení, které provedeme, nám konečně umožní průměrovat:

    \begin{define}\label{LAsqrt}
    \L ukasiewiczovu algebru $(\LL,\wedge,\vee,\otimes,\rightarrow,\sqr,0,1)$ obohacenou o unární operaci \textbf{odmocnina} pro kterou platí:
    \beq
    \sqr(x)\otimes\sqr(x) = x
    \eeq
    nazýváme \textbf{\L ukasiewiczova algebra s odmocninou} \textup{(\LAsq)}.
    \end{define}

    Snadno se přesvědčíme, že
    \beq
    \sqr(x) = \frac{1+x}{2}
    \eeq
    Aritmetický průměr \emph{dvou} hodnot pak můžeme realizovat jako
    \beq
    \sqr(x)\otimes \sqr(y) = \max\left(\frac{1+x}{2}+\frac{1+y}{2}-1,0\right) = \max\left(\frac{x+y}{2},0\right) = \frac{x+y}{2}
    \eeq
    což je vlastně geometrický průměr uvnitř \LAsq. Analogicky by bylo možné definovat libovolnou celočíselnou odmocninu, která by nám umožnila průměrovat libovolný počet hodnot. Filtry založené na průměru mnoha hodnot již však nejsou robustní, a tak se schválně omezíme pouze na druhou odmocninu. I s těmito omezeními je však možné zkonstruovat celou třídu tzv. \emph{dyadických filtrů}, které používají ve váženém průměru zlomky typu $\frac{m}{2^n}$.

    \subsection{Fuzzy-logické funkce}

    Vzhledem k tomu, že při výpočtech výsledků filtrů bude hrát hlavní roli čas, těžko si představit, že například \kk-tou pořadovou charakteristiku budeme implementovat tak, jak je popsána v definici~\ref{theo k-tý prvek}. Přijměme tedy úmluvu, že matematický model nám řekne, jaké operace (funkce,filtry) jsou přípustné, ale při implementaci budeme chtít, aby \emph{pouze výsledek} (nikoliv samotný algoritmus) byl v souladu s teorií, čímž získáme prostor pro optimalizaci kódu na rychlost. Pro další postup bude tedy vhodné rozlišit, které funkce do našeho modelu patří, a které nikoliv.

    \begin{define}\label{def FLF}
    Buď $x \in \LL$ proměnná, $a \in \LL$ konstanta. Pak množinu fuzzy-logických funkcí \textup{(FLF)}, jakožto množinu zobrazení $\LL^k \rightarrow \LL, k \in \mathbb{N}_0$, definujeme indukčním pravidlem:
    \[
    \mathrm{FLF} = x \mid a \mid \f \wedge \g \mid \f \vee \g \mid \f \otimes \g \mid \f \rightarrow \g \mid \sqr(\f)
    \]
    kde $\f,\g$ jsou již \textup{FLF}.
    \end{define}

    \subsection{Používané operátory}\label{operátory}

    Následuje tabulka běžných fuzzy-logických operátorů včetně definic a výsledných hodnot v \LAsq. Všechny operátory vzniknou pouze složením základních operátorů a jsou tedy FLF.

    \begin{table}[h]
    \begin{center}
    \begin{tabular}{llll}
      \toprule
      Operace & Název & Odvození & Význam v \LAsq \\
      \midrule
      $x \vee y$            & spojení                   & definováno & $\max(x,y)$ \\
      $x \wedge y$          & průsek                    & definováno & $\min(x,y)$ \\
      $x \otimes y$         & součin (t-norma)          & definováno & $\max(x+y-1,0)$ \\
      $x \rightarrow y$     & residuum                  & definováno & $\min(1-x+y,1)$ \\
      $ \neg x$             & negace                    & $x \rightarrow 0$ & $1-x$ \\
      $x \leftrightarrow y$ & ekvivalence (biresiduum)  & $(x \rightarrow y) \wedge (y \rightarrow x)$ & $1-|x-y|$ \\
      $x \circ y$           & vzdálenost                & $\neg (x \leftrightarrow y)$ & $|x-y|$ \\
      $x \oplus y$          & součet                    & $\neg (\neg x \otimes \neg y)$ & $\min(x+y,1)$ \\
      $x \ominus y$         & rozdíl                    & $x \otimes \neg y$ & $\max(x-y,0)$ \\
      $x \odot n$           & celočíselný násobek       & $\underbrace{x \oplus x \oplus ... \oplus x}_n, n \in \mathbb{N}$ & $\min(nx,1)$ \\
      $x^n$                 & celočíselná mocnina       & $\underbrace{x \otimes x \otimes ... \otimes x}_n, n \in \mathbb{N} $ & $\max(nx-n+1,0)$ \\
      \bottomrule
    \end{tabular}
    \caption{Odvozené operátory v \LAsq}
    \end{center}
    \end{table}

\section{Citlivost}

\begin{define}\label{def citlivost}
  Buď $\varphi : \LL^n \rightarrow \LL$ \textup{FLF} a $\xx,\yy \in \LL^n$. Pak
  \beq
  \lambda_\varphi = \max_{\xx \neq \yy}\frac{\varphi(\xx) \circ \varphi(\yy)}{\sum_{i=1}^n |x_i - y_i|}\label{citlivost}
  \eeq
  nazveme \textbf{citlivostí} funkce $\varphi$.
\end{define}

Pro další úvahy je dobré znát horní meze citliosti běžných fuzzy-logických operátorů. Ty přehledně shrnuje tabulka \ref{tabulka max citlivosti} (kde $\xx \in \LL^n ,n \in \mathbb{N} ,\;\f ,\g : \LL^n \rightarrow \LL $ jsou FLF) a jsou dokázány v \cite{MajerovaPhD}.

\begin{table}[h]
    \begin{center}
    \begin{tabular}{llp{1cm}ll}
      \toprule
      Operace & $\lambda_{\max}$ && Operace & $\lambda_{\max}$\\
      \midrule
      $\f(\xx)$                         & $\lambda_\f$  && $\f(\xx) \vee \g(\xx)$            & $\max(\lambda_\f,\lambda_\g)$  \\
      $\g(\xx)$                         & $\lambda_\g$  && $\f(\xx) \wedge \g(\xx)$          & $\max(\lambda_\f,\lambda_\g)$  \\
      $\f(\xx) \otimes \g(\xx)$         & $\lambda_\f+\lambda_\g$ && $ \neg \f(\xx)$         & $\lambda_\f$    \\
      $\f(\xx) \rightarrow \g(\xx)$     & $\lambda_\f+\lambda_\g$ && $\sqr(\f(\xx))$         & $\lambda_\f/2$ \\
      $\f(\xx) \leftrightarrow \g(\xx)$ & $\lambda_\f+\lambda_\g$ && $\f(\xx) \oplus \g(\xx)$& $\lambda_\f+\lambda_\g$    \\
      $\f(\xx) \circ \g(\xx)$           & $\lambda_\f+\lambda_\g$ &&$\f(\xx) \ominus \g(\xx)$& $\lambda_\f+\lambda_\g$    \\
      $\f(\xx) \odot n$                 & $n\lambda_\f$ &&  $(\f(\xx))^n$                    & $n\lambda_\f$ \\
      \bottomrule
    \end{tabular}
    \caption{Horní meze citlivosti operátorů v \LAsq}
    \end{center}
\end{table}\label{tabulka max citlivosti}

Citlivost $\lambda_\varphi = 0$ znamená, že funkce $\varphi(\xx) = konst, \, \forall \xx \in \LL^n$ (v čitateli \eqref{citlivost} musí být 0 $\forall \xx$), což odpovídá ztátě veškeré informace nesené vektorem $\xx$. Nízkofrekvenční filtry mají citlivost obecně v intervalu $(0,1]$, vysokofrekvenční pak v intervalu $(1,\infty)$. Zjednodušeně řečeno nízká citlivost způsobuje vymizení detailů, zatímco vysoká může mít za následek vznik nových artefaktů v obraze, vše ale záleží na konkrétním filtru. Z tabulky je dále zřejmé, proč jsme zaváděli odmocninu -- ta jako jediná dokáže citlivost snížit, nebo ji spíše při skládání operací udržet v požadovaném intervalu. \note{dokazovat horní mez citlivosti pro  k-tý prvek, nebo to nechat až na filtry}


\section{Reprezentace obrazových dat}

Při psaní demonstračního kódu jsme se zaměřili na zpracování 3D medicínských dat v odstínech šedi. Jejich zpracování je výpočetně náročnější než u 2D, kvůli vyššímu počtu okolních (viz definice~\ref{def okolí}) voxelů\footnote{základní element 3D obrazu, analogie s pixelem ve 2D -- odtud také jeho název \bq \underline{vo}lume-pi\underline{xel}\eq} a absolutní urychlení výpočtu je tak mnohem citelnější. Data budeme chtít zpracovat pomocí FLF, tzn. musíme nejprve přiřadit vstupům funkce hodnoty konkrétních voxelů.

\begin{define}\label{def obraz}
  \textbf{Obraz} o šířce $w$, výšce $h$ a hloubce $d$ ztotožníme s maticí $\PP \in \LL^{w\times h\times d}$, kde $\LL$ je interval zahrnující všechny přípustné intenzity v obraze. Dále voxel $\PP_{i,j,k}, \,i \in \widehat w, \,j \in \widehat h, \,k \in \widehat d$ označíme jako $x_{i,j,k}$.
\end{define}

    \subsection{Lokální zpracování obrazu}\label{lokální zprac}
    % předem definované pixely vs watershed
    Všechny dále popsané filtry zpracovávají obraz lokálně, to znamená, že do intenzity cílového voxelu se promítnou pouze intezity několika přesně definovaných voxelů z okolí zdrojového voxelu. Jako opačný příklad může sloužit složitější segmentační filtr \emph{rozvodí} (watershed, \cite{Charypar}), u kterého nemůžeme předem říci, které voxely budou mít vliv na výsledek, a dokonce ani kolik jich bude.

    \begin{define}\label{def okolí}
      Buď $R\in \mathbb{N}_0, \,x_{i,j,k} \in \PP$. Pak množinu voxelů
      \beq
      \NN = \Big\{ x_{p,q,r} \in \PP \;\Big\vert\; |p-i| \leq R \wedge |q-j| \leq R \wedge |r-k| \leq R \, \, p,q,r \in \mathbb{Z} \Big\}
      \eeq
      nazýváme \textbf{okolím voxelu $x_{i,j,k}$} o poloměru $R$ a $x_{i,j,k}$ nazýváme \textbf{centrální voxel}. Okolí představuje krychli voxelů $(2R+1) \times (2R+1)\times (2R+1)$, kterou můžeme po plátcích a následně po řádcích přeznačit a uspořádat do \textbf{vektoru okolí} $\Nb = (x_1,x_2,...,x_{(2R+1)^3})$.
    \end{define}

    Z předchozí definice je patrné, že bude problém s okraji obrazu, protože pouze centrální voxel musí ležet uvnitř obrazu. Aby nedocházelo k degeneraci okolí, dodefinujeme intenzity voxelů vně obrazu nějakým zvoleným způsobem \cite{MajerovaPhD}: buď konstatní nulou, hodnotou nejbližího krajního voxelu, periodickým opakováním obrazu, nebo jeho ozrcadlením přes okraj (na hranách se zrcadlí dvakrát, v rozích třikrát). Nejvhodnější způsob závisí na mnoha okolnostech -- od volby filtrů až po charakter samotných dat.

    \subsection{Maska}

    Abychom získali větší variabilitu, ohodnotíme každý voxel z okolí vahou:

    \begin{define}\label{def maska}
      \textbf{Maskou} s poloměrem $R$ nazveme krychlové pole $\MM \in \mathbb{N}_0^{(2R+1) \times (2R+1)\times (2R+1)}$ s prvky $m_{i,j,k}$. Číslo
      \beq
      c = \sum_{i,j,k=1}^{2R+1} m_{i,j,k}
      \eeq
      nazýváme \textbf{kapacita} masky. Masku můžeme taktéž po plátcích a následně po řádcích přeznačit a vytvořit \textbf{vektor vah} $\mathrm{W} = (w_1,w_2,...,w_{(2R+1)^3})$.
    \end{define}

    Binární masku v matematické morfologii (viz sekce \ref{Typologie}) označujeme také jako \emph{strukturní element}. Ve většině aplikací se setkáme pouze s relativně malým $R$ (řekněme $R \leq 3$, \cite{MajerovaPhD}), protože s rostoucím poloměrem masky jednak velmi rychle roste náročnost zpracování ($O(R^3)$ ve 3D krát náročnost filtru) a druhak u nízkofrekvnčních filtrů je efektivnější zmenšit celý obraz a použít menší masku (zvláště, pokud je granularita šumu větší než jeden voxel). U vysokofrekvenčních filtrů nemá větší maska smysl vůbec, neboť tím dochází k nežádoucímu rozmazání.

    \subsection{Seznam}

    Nyní můžeme získat \emph{vážený seznam vstupních hodnot} pro zpracování filtrem jednoduchým přiložením středu masky ($m_{R+1,R+1,R+1}$) na centrální voxel okolí $\NN$:

    \begin{define}\label{def vážený seznam}
    Buď $\MM$ maska s kapacitou $c$ a poloměrem $R$, $\mathrm{W}$ její vektor vah, $\Nb$ vektor okolí, $x_m \in \Nb$ a $w_m \in \mathrm{W}$. Pak c-prvkový seznam
    \beq
    \Lw = (\underbrace{x_1,...,x_1}_{w_1},\underbrace{x_2,...,x_2}_{w_2},...,\underbrace{x_{(2R+1)^3},...,x_{(2R+1)^3}}_{w_{(2R+1)^3}})
    \eeq
    nazvame \textbf{váženým seznamem} hodnot z okolí $\NN$.
    \end{define}

    Pro některé filtry se používá ještě takzvaný \emph{Walshův seznam}:

    \begin{define}\label{def Walshův seznam}
    Seznam o $c+1 \choose 2$ prvcích
    \beq
    \WL = \Big( \frac{x_m + x_n}{2} \;\Big\vert\; x_m,x_n \in \Lw, \,m,n \in \widehat{c} \Big)
    \eeq
    nazveme \textbf{Walshův seznam}.
    \end{define}

    $\WL$ obsahuje průměry všech dvojic z $\Lw$, včetně průměrů prvků se stejným indexem, tedy $\Lw \subset \WL$. Všechny prvky $\WL$ jsou FLF, neboť průměr dvou hodnot je FLF.


% MASKA:
% co to je, váhy (opakovaný výběr), tvoří se z ní seznam jako základní prvek k zpracování
% okraje
% váhy mohou jít proti robustnosti, hrání s váhami

\section{Filtry}\label{Filtry}
    % něco se dělá nad jedním nebo někalika seznamy a je realizovatelné pomocí FLF, je filtr
    % omezujeme použití konstant -- ztrácí se robustnost, de facto je to přepoklad o vstupu

    Nyní už máme dostatečné prostředky k tomu, abychom popsali postup filtrace obrazu:
    \begin{define}\label{def filtr}
      \textbf{Filtrací} zdrojového obrazu $\PP$ na výsledný obraz $\QQ$ stejné velikosti pomocí filtru $\varphi \in \mathrm{FLF}$ a $n$ masek $\MM^{(1)},\MM^{(2)},...,\MM^{(n)}$ rozumíme následující postup:
      \begin{enumerate}
      \item z okolí zdrojového voxelu $x_{i,j,k}^{(\PP)}$ vytvoříme pomocí masek $\MM^{(r)}$ vážené seznamy hodnot $\Lw^{(r)} \,\, \forall r \in \widehat n$
      \item pokud to filtr vyžaduje, vytvoříme z $\Lw^{(r)}$ Walshův seznam pro požadovaná $r$
      \item příslušné seznamy použijeme jako vstup pro $\varphi$
      \item výsledek zapíšeme do odpovídajícího cílového voxelu $x_{i,j,k}^{(\QQ)}$
      \end{enumerate}
      \textbf{Filtrem} tedy rozumíme libovolnou funkci $\varphi \in \mathrm{FLF}, \,\varphi : \LL^s \rightarrow \LL$, kde $s$ je součet počtu prvků všech seznamů použitých na vstupu $\varphi$.
    \end{define}

    V našem případě bude vždy $n = 1$, ale mohou se vyskytnout i fitry využívající např. více asymetrických masek.

    \subsection{Typologie filtrů}\label{Typologie}       % (implementační hledisko)

    Hledisek, podle kterých by se daly filtry roztřídit je více. Jedno, souvidející se tím, jaké frekvence se výsledném obraze zachovají, jsme již zmínili. Vzhledem k tomu, že se dále v práci chceme věnovat hlavně implementaci filtrů, zmíníme ještě jedno rozdělení, které určuje, jak filtr zachází se vstupními daty -- totiž rozdělení na \emph{morfologické} a \emph{statisticky motivované} filtry.

        \subsubsection{Morfologické}
        Jak název napovídá, do této kategorie patří filtry snažící se nějak postihnout či zvýraznit charakter tvarů v obraze. Základ tvoří dva \emph{morfologické operátory} \emph{eroze} a \emph{dilatace}:

        \begin{define}\label{de eroze dilatace}
          Buď $\xx$ seznam délky $c$ vytvořený maskou $\MM$. Pak filtr
          \beq
          \EE(\xx) = \bigwedge_{i=1}^c x_i
          \eeq
          nazveme \textbf{erozí} s maskou $\MM$ a filtr
          \beq
          \DD(\xx) = \bigvee_{i=1}^c x_i
          \eeq
          nazveme \textbf{dilatací} s maskou $\MM$.
        \end{define}

        Filtry jsou komplementární, proto popíšeme pouze erozi: původ jejího názvu je nejvíce patrný při aplikaci na binární (černo-bílý) obraz, kde způsobuje zmenšení bílých oblastí, odstranění bílých detailů (světlé části šumu typu \emph{sůl a pepř}) a rozšíření tmavých \bq trhlin\eq. Na obraze v odstínech šedi dochází obecně ke ztmavnutí, které se nejvíce projeví na ostrých hranách, jinak je efekt srovnatelný s černo-bílým protějškem. Oba filtry jsou triviálně FLF, jakékoliv jejich složení tedy také.

        Z povahy operací je zřejmé, že váhy větší než jedna se budou chovat jako váhy rovny jedné. Proto stačí uvažovat binární masku (\emph{strukturní element}). To je obecná vlastnost i dalších morfologických filtrů, které nejsou postavené přímo na erozi a dilataci, protože jejich výsledky mají vždy charakter maxima nebo minima ze seznamu hodnot.

        Další oblíbené morfologické operátory vzniknou právě kombinací těchto dvou:

        \begin{define}\label{de eroze dilatace}
          Buď $\PP$ obraz. Pak filtr
          \beq
          \OO(\PP) = \DD(\EE(\PP))
          \eeq
          nazveme \textbf{otevřením} a filtr
          \beq
          \CC(\PP) = \EE(\DD(\PP))
          \eeq
          nazveme \textbf{uzavřením}. Zápisem $\EE(\PP)$ rozumíme provedení $\EE(\xx)$ na všechny pixely obrazu $\PP$. U každého filtru jsou eroze i dilatace prováděny se stejnou maskou $\MM$.
        \end{define}

        Filtry odstraní světlé (respektive tmavé) detaily, aniž by zbytek obrázku příliš poškodily, navíc platí:

        \beq
        \OO(\OO(\PP)) = \OO(\PP), \quad \CC(\CC(\PP)) = \CC(\PP)
        \eeq

        Jejich kombinací můžeme získat dva jednoduché vyhlazovací filtry $\OO(\CC(\PP))$ a $\CC(\OO(\PP))$.

        Rozdílem $\DD(\xx) \ominus \EE(\xx)$ získáme jednoduchý detektor hran (je FLF, neboť $\ominus,\DD,\EE$ jsou FLF), nazývaný také \emph{Minkowského kolása} (pro podobnost klobásy a výsledku filtrace binárního obrazu pomocí kruhové masky s velkým poloměrem). Další fuzzy hranové detektory můžeme nalézt např. v \cite{Bělíček}. U detektorů hran je dobře vidět velká citlivost morfologických filtrů na konkréntní podobu strukturního elementu; jeho změnou můžem například (při zachování jeho kapacity) zařídit, že budou detekovány hrany jdoucí pouze v jednom směru, což zcela mění charakter filtru.

            % zmínit binární masku jako SE
        \subsubsection{Statisticky motivované filtry}\label{statisticky motivované}

        Statistické filtry pracují nad \emph{seřazeným} seznamem $\Lw$, nebo $\WL$ a využívají různé kombinace k-té pořadové statistiky. Používají se především k odstaňování šumu. Vyšší váha pixelu v masce znamená, že se výběrem k-tého prvku do jeho hodnoty trefíme pravděpodobněji, avšak přílišné zvýhodnění pixelu ubere filtru robustnost, neboť pak hůře odfiltruje okrajové hodnoty. Asi nejznámějším filtrem této kategorie je \emph{medián}.

        \begin{define}\label{def median}
          Buď $\xx$ seřazený seznam délky $c$ vytvořený maskou $\MM$ s nejmenším prvkem $x_1$ a největším prvkem $x_c$. Pak filtr
          \beq
          \MED(\xx) = \frac{1}{2}\Big(x_{\lfloor \frac{c+1}{2} \rfloor}+x_{\lceil \frac{c+1}{2} \rceil}\Big)
          \eeq
          nazýváme \textbf{medián}.
        \end{define}

        Medián, jako klasický robustní filtr, výborně odstraňuje šum typu \bq sůl a pepř\eq. Typickým šumem, kde se výhodně používají robustní filtry (alespoň v první fázi filtrace), je kontaminovaný gaussovský šum. Kontaminace znamená, že část dat je zničena (nahrazena šumem \bq sůl a pepř\eq, nebo jinými hodnotami na hranici rozsahu), což se může být způsobeno například kosmickým zářením prolétnuvším detektorem. Při zničení $n$ \% obrazu mluvíme o $n$\% kontaminaci. Klasické průměrování by zde sice odstranilo (zmenšením variance, vyhlazením) gaussovský šum, ale zahrnutím zničených dat do průměru by výsledek degradoval. Filtr tedy musí nejprve vybrat pouze relevatní data (odolnost proti kontaminaci\footnote{můžeme brát jako míru robustnosti}), a až poté může být průměrováním odstraněn gaussovský šum. Medián je odolný limitně proti 50\% kontaminaci, ale při lichém počtu prvků vybírá jen jeden prvek, a tudíž vůbec nesnižuje varianci šumu. Toto chování můžeme na úkor zmenšení odolnosti vůči kontaminaci zlepšit pomocí průměrování více vstupů:

        \begin{define}\label{def kvazimedian}
          Buď $\xx$ seřazený seznam délky $c$ vytvořený maskou $\MM$ s nejmenším prvkem $x_1$ a největším prvkem $x_c$, $q\in \mathbb{N}$, $1 \leq q \le \lfloor \frac{c+1}{2} \rfloor-1$. Pak filtr
          \beq
          \MED_{q}(\xx) = \frac{1}{2}\Big(x_{(\lfloor \frac{c+1}{2} \rfloor-q)}+x_{(\lceil \frac{c+1}{2} \rceil-q)}\Big)
          \eeq
          nazýváme \textbf{kvazi-medián} s parametrem q.
        \end{define}

         \begin{define}\label{def BES}
          Buď $\xx$ seřazený seznam délky $c$ vytvořený maskou $\MM$ s nejmenším prvkem $x_1$ a největším prvkem $x_c$. Pak filtr
          \beq
          \BES(\xx) = \frac{1}{4}\Big(x_{\lceil \frac{c}{4} \rceil}+x_{\lfloor \frac{c+1}{2} \rfloor}+x_{\lceil \frac{c+1}{2} \rceil}+x_{\lfloor \frac{3c+4}{4} \rfloor}\Big)
          \eeq
          nazýváme \textbf{Best Easy Systematic} estimator.
        \end{define}

        Filtry jsou FLF, neboť \kk-tý prvek je FLF a aritmetický průměr také. Kvazi-medián průměruje vždy dva prvky, tzn. variance gaussovského šumu se snižuje na polovinu a odolnost vůči kontaminaci klesá s $q$ až k nule -- to je důvod, proč se většinou $q$ volí relativně malé. BES naproti tomu má pevnou limitní odolnost proti kontaminaci 25 \% (vše se vejde pod první, nebo nad třetí kvartil) a varianci snižuje 3krát až 4krát, opět podle toho, zda je počet prvků lichý, nebo sudý.

        Dalšího snížení variance při zachování odolnosti vůči kontaminaci dosáhneme použitím Walshova seznamu. Díky tomu se může počet průměrovaných vstupů až zdvojnásobit a variance tak klesne na polovinu oproti filtrům bez Walshova seznamu. Toto je další důvod, proč i v této kategorii filtrů není vhodné používat jinou, než binární masku -- při vyšší váze hrozí nebezpečí, že ten samý prvek bude vybrán několikrát a průměrováním jeho samého se sebou se variance šumu nesníží. Mediánu nad Walshovým seznamem říkáme Hodges-Lehmannův medián, BES nad Walshovým seznamem značíme WBES. Diskuse snižování variance a odolnosti proti kontaminaci je provedena v sekci \ref{diskuse masky}.

    \subsection{Sítě a kompromisní filtrování}\label{sítě}

        Další možnost, jak zlepšit vlastnosti fitrů (zejména v oblasti odstanění šumu), je kompromisní filtrování a konstrukce sítí filtrů \cite{Compromise denoise},\cite{Minmax denoise}. Přístup spočívá v tom, že na obraz použijeme celou škálu oblíbených (třeba i průměrovacíc filtrů) filtrů a jejich výsedky zpracujeme pomocí další FLF $\phi$ na konečný výsledek (případně na několik mezivýsledků a ty pak stejným způsobem zpracujeme na konečný výsledek). Hledání $\phi$ se děje podobným způsobem, jako učení neuronové sítě, případně experimentováním. To je v obou případěch velmi výpočetně náročné, což je také jeden z důvodů vývoje paralelních algoritmů pro tyto filtry.


% Kvalita obrázku:

% TODO: viziualizovat rozdíly mezi filtry, jako "standardy" brát výsledky předchozích prací 

    \chapter{implementace na CPU}
        % Celočíselná aritmetika
            % Přesnost
        % Uložení obrazových dat v paměti (okraje)
            % (něco k formátu .hdr?)

    % algoritmy -- co patří k CPU do CPU, co ke GPU do GPU

    \chapter{Výpočty na GPU}
        % Vývoj GPU
        % Programovací prostředí
            % CUDA
        % Architektura GPU (NVIDIA)
            % Hierarchie paměti
            % Hierarchie paralelizace

        % -*-coding: utf-8 -*-

% Vývoj GPU
% Programovací prostředí
    % CUDA
% Architektura GPU (NVIDIA)
    % Hierarchie paměti
    % Hierarchie paralelizace


%čím rychlejší chceme výpočty, tím více omezneí musíme splnit

%Zatím to vypadá, že klasické qsorty atd. budou mít velké overheady
%Zkusit qsort po warpech a porovnat s tím O(n2), co mám teď, ten taky zjemnit na warpy
%Obecně naplnit sdílenou paměť, něco jako pár threadů na pixel

%    -- je to porovnání s 350 na jeden krok VS porovnání s 32 v jednom z několika kroků + scan sumy

%BES na CPU (rozkrájený quickselect)

V této kapitole se stručně seznámíme s historií a vývojem v oboru výpočtů na GPU, rozebereme vlastnosti, přednosti a nevýhody GPU architektury v porovnání s klasickým CPU a nakonec popíšeme programovací model CUDA\footnote{Compute Unified Device Architecture} od společnosti Nvidia, který použijeme pro implementaci filtrů na GPU.

\section{Vývoj GPU}

    kolik sem dát historie? Už to považovat za zajetý obor, nebo zmiňovat celý vývoj? -- Zhruba odstaveček na vývoj od grafického API až po CUDA

    S použitím GPU pro jiné, než grafické výpočty se začalo experimentovat, jakmile přestaly být grafické karty -- hlavně díky rozvoji herního průmyslu -- pouhým jednoúčelovým zařízením a staly se z nich (alespoň částečně) programovatelné paralelní procesory. Vzhledem k tomu, že karty byly primárně ke zpracování grafiky, daly se výpočty provádět pouze pomocí grafického API například přes textury a programovatelné pixel-shadery, což značně snižovalo efektivitu zpracování díky vysoké režii API.

    Zřejmě první velká společnost, která se rozhodla vyjít vstříc požadavkům na konstrukci GPU jako univerzálně použitelné vysoce paralelní výpočetní jednotky, byla Nvidia, když změnila celý jeho koncept nově vyvíjeného čipu a zavedla tzv. CUDA-jádra (obecnější a pružnější, než jednotky dedikované ke specifickým grafickým operacím\footnote{ty jsou ovšem na kartách stále přítomny a lze je v programu pro tyto specifické operace použít (např. výpočet $\sin$, $\cos$)}). V únoru 2007 pak představila první verzi vývojového nástroje CUDA, který umožňoval efektivně využívat hardware GPU pomocí několika rozšíření jazyka C a posléze \Cpp. CUDA-jádra dokonce umí počítat nativně v dvojnásobné přesnosti; u běžných karet jsou ale tři čtvrtiny jader pro dvojitou přesnost deaktivovány \cite{Heller} (pro grafické operace nejsou třeba), a pro plný výkon ve dvojité přesnosti si musíme koupit (řádově dražší) kartu ze série TESLA, která je primárně určena pro složité výpočty. Veškerý vývojový software poskytuje Nvidia zdarma.

    V současnosti existují k CUDA dvě alternativy: prostředí OpenCL\footnote{Open Compute Language}, zaštítěné sdružením Khronos Group, které se profiluje jako standard pro heterogenní paralelní programování a DirectCompute od Microsoftu stojící na balíku DirectX verze 10 a vyšší. Výhodou OpenCL je, že stejný kód lze zkompilovat jak pro CPU, tak pro GPU výrobců ATI a Nvidia\footnote{jinak se bohužel jedná o vzájemně nekompatibilní technologie}. Z principu tedy nemůže poskytovat tak pohodlný přístup, jako CUDA a výsledkem je poměrně rozsáhlý kód. \emph{Dále se budeme zabývat pouze prostředím CUDA a hardwarem s ním souvisejícím.}

    \subsection{Současnost}

     V současnosti je GPGPU\footnote{General Purpose computations on Graphics Processing Units} již etablovaný obor. Hlavní trend vývoje GPU je nyní podobný, jako v počátcích CPU -- spočívá v odstraňování omezení, která musíme na kód klást, abychom dosáhli optimálního výkonu a větší univerzalizaci hardwaru. Tím pádem se i rozšiřuje množina úkolů vhodných pro zpracování na GPU. Jednotlivé výpočetní jednotky na kartách už dávno nejsou pouhými jednoduchými vektorovými procesory, ale stále více se blíží plnohodnotnému (vícejádrovému) procesoru, i když si samozřejmě svůj vektorový charakter zachovávají. Asi nejvíce je to vidět na práci s pamětí, kde u nejnovějšího čipu \FERMI z dílny Nvidia přibyla vrstva chache, čímž došlo k rozvolnění přístupu do (největší) globální paměti, ale došlo i k rozdělení vektorového procesoru na dva samostatně fungující díly.

\section{Architektura GPU}

    \subsection{Rozdíly CPU a GPU}

        Nyní se podrobněji podíváme na specifika architektury čipů grafických karet. Hlavní otázkou je, pro jaké úlohy jsou vůbec GPU vhodné. Obrázek~\ref{cpu vs gpu} zhruba ukazuje, jak velká část čipu (DRAM ovšem není přímo na čipu) CPU a GPU je dedikována pro určitý druh operací.

        \begin{figure}[h]\label{cpu vs gpu}
          \includegraphics[width = \textwidth]{src/2Gpu/CPUGPU.png}
          \caption{Rozdíly využití čipu na CPU a GPU}
        \end{figure}

        Vidíme, že velkou část čipu CPU zabírá cache a kontrolní logika. Ty zajišťují několika ALU\footnote{Arithmetic Logic Unit} dostatečný přísun dat a instrukcí (např. pomocí hyper-threadingu a branch-prediction), lhostejno jak moc se program větví, nebo jak jsou data uspořádána v DRAM.

        GPU se naproti tomu skládá z několika samostatně funkčních vektorových procesorů -- tzv. SM\footnote{Streaming Multiprocesor} jednotek, z nichž každá má vlastní malou cache a z kontrolní logiky má naprosté minimum. Z toho plyne jednak jistá nutná disciplína při přístupu do DRAM, která je ale narozdíl od té na CPU lépe optimalizovaná na sekvenční čtení, a za druhé musíme běh programu přizpůsobit tomu, že GPU je po částech SIMD\footnote{Single Instruction Multiple Data}, respektive SIMT\footnote{Single Instruction Multiple Thread -- název pocházející od Nvidia} -- narozdíl od vícejádrového CPU, které je plnohodnotné MIMD\footnote{Multiple Instruction Multiple Data}. Podrobný popis nejnovějšího SM z čipu \FERMI lze nalézt v \cite{Fermi}.

        Obecně můžeme říci (protože velká část čipu GPU je dedikována pro aritmetiku), že GPU je nejvíce vhodná pro aritmeticky intenzivní výpočty, tzn. mající vysoký poměr počtu aritmetických operací ku počtu přístupů do paměti.

    \subsection{Algoritmy vhodné pro GPU}

        Před paralelizací algoritmu pro GPU musíme tedy zvážit, zda to má vůbec smysl -- pro dosažení optimálního urychlení musí algoritmus v co největší míře splňovat následující body:
        \begin{itemize}
          \item rozložitelnost výpočtů na nezávislé části, které lze vykonávat paralelně (výsledek jedné nezávisí na výsledku předchozí)
          \item malý počet přístupů do paměti, nejlépe sekvenční/blokové
          \item podobný kód ve všech paralelních částech, minimální a předvídatelné větvení programu
        \end{itemize}

        Jak uvidíme, filtrování obrazu nesplňuje tyto body úplně (např. přístupy do paměti) -- kdyby splňovalo vše, tak se jím nemá příliš cenu zabývat -- ale dostatečně na to, abychom dosáhli použitelných výsledků.

\section{Pogramovací model CUDA a jeho HW implementace}

    \subsection{Úvod}
    Pro maximální efektivitu paralelizace vyvinula Nvidia několikavrstvou dobře škálovatelnou strukturu s přímou návazností na svůj hardware. Nejmenší výpočetní jednotkou z pohledu programu je \emph{thread} (vlákno), který fyzicky běží na jednom CUDA-jádře v SM. Thready se sdružují do \emph{thread-bloků}, které běží na celém SM a mohou mít 1D, 2D, nebo 3D strukturu, podle toho jakého charakteru jsou zpracovávaná data. Poslední článek tvoří \emph{grid} běžící na celé GPU obsahující (zatím) nejvýše 2D strukturu thread-bloků. Z programu pak výpočty spouštíme pomocí volání \emph{kernelu} obsahujícího náš kód, kterému sdělíme v kolika thread-blocích má běžet, jakou mají mít strukturu v gridu, kolik threadů má být uvnitř thread-blocku a jak mají být uspořádány. To umožňuje ten samý kernel spouštět na různých zařízeních, protože sám hardware rozhoduje, jak jednotlivé výpočetní elementy mezi SM (a posléze CUDA-jádra) rozdistribuuje.

    \subsubsection{Úrovně přístupu}

    Na straně uživatele poskytuje Nvidia dvě rozhraní: nízkoúrovňové CUDA Driver API a vysokoúrovňové CUDA runtime API, v současnosti ve verzi 4.0. Driver API poskytne uživateli totální kontrolu nad kartou výměnou za složitější a delší kód -- uživatel musí zajistit inicializaci zařízení, přesun dat a funkčních parametrů pro kernel na kartu a přesun a spuštění kernelu samotného\note{ kernel musí být v extra souboru?}. Rozhraní Driver API je v C a funkčně zhruba odpovídá OpenCL. Runtime API je rozšířením C, které umožňuje kernel napsat podobně jako funkci v C a při volání jí speciální syntaxí sdělit, kolik threadů a bloků chceme spustit. O inicializaci a vše ostatní se postará runtime. Nevýhodou Runtime API je, že jeden CPU proces nemůže používat více zařízení, což se ale nové verze snaží vylepšit.

    Mixování obou přístupů (např. právě kvůli použití více zařízení) se nedoporučuje, ale v rozumné míře možné je -- pravidla mixování stanovila až CUDA 4.0. V našem kódu používáme výhradně Runtime API, proto se dále budeme zabývat pouze jí.

    \subsubsection{Výpočetní schopnost}

    Tento termín bude dále v textu často používán. Výpočetní schopnost\footnote{Compute Capability} (VS) udává fyzickou verzi čipu -- všechna CUDA-enabled zařízení před před architekturou \FERMI mají VS 1.x, čipy s touto architekturou 2.x. Různé VS se liší množsvím paměti, registrů, strukturou SM atd. Největší skok je samozřejmě mezi 1.x a 2.x. Podrobný popis rozdílů mezi VS je k dispozici v \cite[přílohy]{CUDA programming g.}. Pokud budeme dále uvádět konkrétní hodnoty, budou pro VS 1.1, pro kterou je kód primárně optimalizován. Pokud to bude důležité, bude v závorce uvedena hodnota pro VS 2.0 -- na kartách s těmito VS budou totiž i počítány experimentální výsledky.

    \subsection{Kernel}

    Runtime API nám dovoluje míchat kód pro CPU i GPU v jednom souboru, definici kernelu proto rozlišíme specifikátorem
    \Vr"\cy{__global__}" (příklad převzat z \cite{CUDA programming g.}).

    \begin{Verbatim}[commandchars = \\\{\}]
    // definice kernelu
\cy{__global__} \bl{void} MyVecAdd(\bl{float} *A, \bl{float} *B, \bl{float} *C)\{
    \bl{int} i = \cy{threadIdx}.x;
    C[i] = A[i] + B[i];
\}
    // volání kernelu v programu
    MyVecAdd<<<1,N>>>(A,B,C);
    \end{Verbatim}

    Při volání v příkladu specikujeme, že chceme jeden thread-blok a N threadů jako vektor, vektory A, B a C musí být při volání již připraveny v paměti karty. Po spuštění dostane každý thread (potažmo thread-blok) své ID, které je v kódu přístupné skrze read-only proměnnou \Vr"\cy{threadIdx}" (potažmo \Vr"\cy{blockIdx}") a umožní tak diferenciaci threadů. V kernelu tedy píšeme kód \emph{jediného threadu}, jehož konkrétní chování se po spuštění \emph{odvíjí od přiděleného ID námi definovaným způsobem}.

    \subsection{Hardwarová paralelizace, multithreading}

    Zde popíšeme, co se skutečně děje při spuštění kernelu na GPU. Přitom se omezíme pouze na jeden SM, protože všechny SM na kartě pracují zcela nezávisle a jakákoliv jejich synchronizace v rámci jednoho kernelu není možná.

    Po spuštění kernelu se bloky rozdistribuují mezi jednotlivé SM. Kolik bloků se vejde současně na jeden SM je dáno tím, kolik registrů a sdílené paměti blok spotřebuje -- tyto hodnoty jsou pevně stanoveny při spuštění kernelu. Pokud se na SM nevejde ani jeden blok, spouštění kernelu selže. V opačném případě hardware umístí na každý SM maximální možný počet bloků, zbývající bloky budou čekat na uvolnění některého SM.

    SM má však pouze 32 výpočetních jednotek, tzn. bloky běží současně pouze virtuálně. Důvod tohoto \bq přesycení\eq~ spočívá v \emph{zakrytí latencí} způsobených například čtením a zápisem do paměti: pokud například nějaký \emph{warp} (což je oněch 32 současně běžících threadů) čeká na načtení operandů z paměti, může mezitím jiný warp (protože má např. proměnné připravené v registrech) provádět výpočty -- i když je to třeba warp z jiného bloku. Rychlé přepnutí warpů\footnote{tzv. context switch} je umožněno tím, že thread na GPU je narozdíl od toho na CPU velmi jednoduchá struktura a multithreading na GPU je dělán hardwarově. Z toho vyplývá jisté omezení na maximální počet bloků a threadů/SM (přesná čísla lze nalézt v \cite{CUDA programming g.}). Ve většině případů ovšem dříve vyčerpáme registry, nebo sdílenou paměť, než se dostaneme k tomuto stropu.

    Pro efektivní využití výpočetních jednotek je nutné, aby pokud možno všechny thready v jednom warpu prováděly stejný kód, neboť celý warp zvládá naráz vykonávat pouze jedinou instrukci (dvě instrukce pro dva 16threadové půlwarpy pro VS 2.x), což odpovídá filozofii SIMD. Všechna větvení kódu by tedy měla být směřována na hranice warpů, tzn. mezi 32 po sobě jdoucích threadů. V opačném případě bude warp rozdělen na části se stejnou posloupností instrukcí a ty budou zpracovány sériově.

    \subsection{Práce s pamětí}

    Jak jsme zmínili v úvodu kapitoly, paměťové zdroje karet jsou poměrně omezené a vhodné zacházení s pamětí je tak prvním předpokladem pro to, aby paralelizovaný algoritmus dosáhl optimálních výsledků. Vývoj jde ale v této oblasti hodně dopředu a architektura FERMI má již paměti mnohem více a lépe strukturované.

    \subsubsection{Registry}

    Jsou velmi podobné klasickým registrům na CPU až na to, že GPU obsahuje pouze 32-bitové registry a nemá žádné dedikované pro dvojitou přesnost \note{?}. Jsou zdaleka nejrychlejší pamětí dostupnou pro thread a jsou do nich ukládány všechny lokální proměnné, kromě polí. Karty s VS 1.x mají 8192 a více registrů/SM, VS 2.x má 32768 registrů/SM. V kódu není třeba je nějak značit, většina lokáních proměnných (viz dále) se do nich sama uloží.

    \subsubsection{Sdílená paměť}

    Jedná se o paměť zabudovanou v každém SM, dostupnou na úrovni bloku a umožňující komunikaci threadů v bloku. Jako de facto manuální (a pro VS 2.x i částečně automatická) cache je stále na čipu GPU a tudíž je poměrně rychlá. Pro optimální rychlost je rozdělena do 32 banků tak, že thready (jednoho warpu) čtoucí 32 po sobě jdoucích položek z pole v této paměti přistupují každý do jednoho banku a celé čtení je možné vyřídit jako jedinou operaci.

    V kódu statické pole pevné délky ve sdílené paměti deklarujeme pomocí identifikátoru \Vr"\cy{__shared__}" -- takto je definováno jedno pole pro celý blok (obyčejná proměnná bez identifikátoru by byla vytvořena pro každý thread) a všechny thready bloku do něj mají přístup. Synchronizace čtení a zápisu je ponechána na uživateli.

    Pole ve sdílené paměti je možné alokovat i dynamicky z CPU: v kernelu ho deklarujeme pomocí identifikátoru \Vr"\bl{extern}":
    \begin{Verbatim}[commandchars = \\\{\}]
\bl{extern} \cy{__shared__} \bl{datovýtyp} jménopole[];
    \end{Verbatim}
    Při spouštení kernelu poté pomocí třetího parametru {\tt Ns} zadáme, kolik chceme navíc (kromě režie) alokovat bytů sdílené paměti pro blok.
    \begin{Verbatim}[commandchars = \\\{\}]
MyKernel<<<1,N,Ns>>>(parametry);
    \end{Verbatim}
    Námi deklarované pole se naváže na začátek této alokované paměti. Nevýhodou je, že při deklaraci více dynamicky alokovaných polí budou všechny začínat na začátku této paměti (jako v unii) a případné rozdělení pole musíme tedy provést ručně pomocí ukazatelů a offsetů. Dynamická alokace přímo z kernelu není (principiálně) možná. Životnost dat ve sdílené paměti je stejná jako životnost bloku.

    \subsubsection{Globální paměť}

    Největší paměť na GPU (jednotky GB). Protože není fyzicky umístěna na čipu, přístup do ní je velmi pomalý -- řádově stovky taktů -- nicméně je optimalizová pro sekvenční čtení a zápis z GPU. Hardware totiž umožňuje přístup pouze po 128-bitových blocích. Pokud nechceme, aby došlo ke zpomalení, je nutné, aby všechny thready v jednom warpu (a tudíž vykonávající stejnou instrukci) přistupovaly do co nejmenšího počtu těchto bloků, ideálně do jednoho. Tím dojde k tzv. \emph{sdruženému přístupu}\footnote{Coalesced access} a všechny přístupy threadů do sousedících míst v paměti mohou být obslouženy v rámci jediného požadavku. Konkrétní způsob (omezení), jak mohou thredy k paměti v bloku přistupvat je závislý na výpočetní schopnosti karty a jeho popis lze nalézt v \cite{CUDA programming g.}.

    Alokace, čtení a zápis do této paměti je možný i z CPU (samozřejmě na jiné bázi), data zapsaná tímto způsobem mají pak životnost stejnou, jako aplikace.

    V kódu se ukazatel do globální paměti neliší od jiných ukazatelů\footnote{K dispozici je i \bq novější\eq ~obecný typ ukazatele do globální paměti {\tt DevicePtr*}, který vyžadují nekteré fukce. Spíše jde ale o snahu nějak data na CPU a GPU formálně oddělit. Kód využívající tento typ ukazatele se nám však nepodařilo zprovoznit. Se starší verzí jsou však dobré zkušenosti.}, postup zkopírování dat na GPU je následující:
    \begin{enumerate}
      \item Vytvoření běžného ukazatele na požadovaný typ
      \item Přiřazení ukazatele k alokovanému poli v globální paměti pomocí {\tt CudaMalloc(...)}
      \item Zkopírování dat z pole v RAM na GPU pomocí {\tt CudaMemcpy(...)}
      \item Ukazatel se předá kernelu přes obyčejný parametr
    \end{enumerate}
    Na CPU nelze ukazatel do globální paměti dereferncovat přímo, na GPU to však lze bez jakýchkoliv omezení.

    Samotnou alokaci opět není možné provádět z GPU, pokud však využívá thread větší statické pole jako lokální proměnnou, nebo tolik proměnných, že se nevejdou do registrů\footnote{tzv. Register spilling}, uloží se přebytek v \emph{lokální paměti}. Lokální zde znamená pouze omezení životnosti dat, data jsou ve skutečnosti uložena do globální paměti -- se všemi nevýhdami, které z toho plynou. Z tohoto důvodu je dobré se použití lokální paměti jakýmkoliv způsobem vyhnout.

    \subsubsection{Konstantní paměť}

    Konstatntní paměť je 64 KB velká část globální paměti s vlastní cache, tudíž je celkem rychlá a vhodná pro objemné konstanty. Je přístupná z CPU pro čtení i zápis, dynamická alokace není možná ani z CPU. Z GPU lze paměť pouze číst -- proto konstatní.

    V kódu se pro tuto paměť používá identifikátor \Vr"\cy{__constant__}" a proměnné s tímto identifikátorem musíme chápat spíše jen jako pojmenování místa v konstantní paměti karty. Poněkud nepříjemné je, že z tohoto důvodu musí být všechny tyto proměnné v kódu definovány jako globální a na úrovni souboru. Nesmí být tedy schovány do jakéhokoliv prostoru jmen, ani do třídy (byť jako statické), jak je zvykem ve slušném \Cpp kódu. Hodnoty do proměnných (polí) v konstantní paměti přesuneme z RAM pomocí {\tt CudaMemcpyToSymbol(...)}.

    \subsubsection{Texturová cache}

    Texturová paměť někdy představuje rychlejší alternativu k čtení (\emph{nikoliv zápisu}) z globální paměti. Část dat můžeme totiž prohlásit za \emph{texturu}, což znamená, že data budou každým SM částečně cacheována (pokud je textura větší než cache). Velikost cache je 6-8 KB/SM, přitom hardware sám odhaduje, která data bude chtít program načíst a ty do ní přesune\footnote{tzv. Texture fetching}. Pokud jsou požadována data, která nejsou v cache k dispozici\footnote{tzv. Cache miss (v opačném případě Cache hit)}, jsou normálním způsobem načtena z globální paměti. Používání textur je tedy výhodné, pokud bloky potřebují relativně malé a dobře definované kousky dat, které se do cache vejdou celé.

    Protože se jedná o činnost hodně blízkou klasické grafice, je v kódu citelně znát odkaz grafického API. Před použitím textury musíme definovat referenci na texturu (opět globální stejně jako konstantní proměnné) a ještě na CPU ji svázat s požadovanými daty pomocí {\tt cudaBindTexture(...)}. Obě operace potřebují značné množství parametrů (potřebných při používání textur pro původní účel), jejich popis lze nalézt v~\cite{CUDA programming g.}. Číst z takto připravené textury je na GPU možné pomocí funkce {\tt texXDfetch(...)} ({\tt X} je 1,2,3).

    VS 2.x již umožňuje volitelné cacheování globální paměti za použití části sdílené paměti, textury ale stále představují průchozí alternativu. Nezaberou totiž sdílenou paměť a využijí několik cenných KB paměti jinak nepoužívané.

    \subsection{Možnosti synchronizace}

    Synchronizace threadů je nutná pro jakoukoliv jejich kooperaci mezi sebou. CUDA umožňuje synchronizaci na úrovni bloku pomocí instrukce \Vr"\cy{__syncthreads()}", která zastaví běh threadů, dokud všechny v jednom bloku tuto instrukci nezavolají. To se hodí například při práci se sdílenou pamětí, kdy chceme zajistit, aby z ní všechny thready začaly číst až poté, co do ní budou (např. menším počtem threadů) zapsána relevantní data.

    CUDA navíc poskytuje i několik speciálních instrukcí čistě pro synchronizaci přístupů do paměti (viz \cite[přílohy]{CUDA programming g.}), což je jediná možnost, jak částečně synchronizovat thready v celé aplikaci -- ovšem za cenu výrazného zpomalení (poruší se přepínání bloků na SM kvůli překrývání latencí).

% v implementaci chceme už jen popisovat optimalizace, programming model už musí být hotový 

    \chapter{Třídící algoritmy} % možná před GPU
        % Quickselect
        % Zapomětlivé třídění

        

\section{Quickselect}

bla

\section{Zapom�tliv� t��d�n�}

bla

    \chapter{Implementace v CUDA}
        % Kernel
        % Paměťové optimalizace
        % Instrukční optimalizace
        % ??

        % -*-coding: utf-8 -*-

Před tím, než se podíváme na implementaci konkrétních filtrů, probereme některé optimalizace, které budeme často používat. Mnoho z nich je velmi obecných a používají se pro jakýkoliv kód spuštěný na GPU. Naopak některé optimalizace popisované v \cite{CUDA programming g.,CUDA best practices} v našem kódu nepoužijeme, neboť pro nás nejsou vhodné. Pro jejich podrobnější popis odkazujeme čtenáře na zmíněnou literaturu.

\section{Použité optimalizace}

    Dvě hlavní třídy optimalizací se týkají optimalizace práce s pamětí a toku instrukcí. V našem případě (a částečně obecně) má vyšší prioritu optimalizace paměti. Přístup do paměti totiž vykazuje narozdíl od vykonávání instrukcí vysokou latenci a náš kód do ní potřebuje přistupovat, v poměru k počtu aritmetických operací, velmi často\footnote{kód je tzv. memory-bound}.

    \subsection{Optimalizace paměti}

        Následující přehled ukazuje různé běžné paměťové operace seřazené od nejpomalejších:
    \begin{itemize}
      \item Přesun dat z CPU na GPU a zpět
      \item Čtení a zápis do globální paměti GPU
      \item Čtení z texturové cache a dalších cache
      \item Čtení a zápis do sdílené paměti
      \item Čtení a zápis do registrů
    \end{itemize}

        \subsubsection{Tok dat mezi CPU a GPU}

        Pro jeho urychlení slouží celá řada typů RAM optimalizovaných pro konkrétní typy operací (CPU pouze zapíše, GPU pouze čte apod.), stále se však jedná o komunikaci přes PCI Express a ta bude proti komunikaci uvnitř karty vždy řádově pomalejší. Nejefektivnější, a pro náš případ nejproveditelnější optimalizací, je redukce těchto přenosů na nezbytné minimum, což není velký problém -- zpracovávaný obraz (obrazy) stačí na začátku přesunout na GPU, všechny mezivýsledky, které nepotřebujeme, ukládat tamtéž a pouze výsledek poslat zpět.

        \subsubsection{Čtení z globální paměti}\label{globální pam opt}

        V našem případě sice čteme souvislé bloky paměti, avšak díky závislosti na uživatelské volbě masky a rozměrech obrázku je nemožné zajistit, aby byl čtený blok \emph{zarovnán} na 128 bytů, jak to vyžaduje CC 1.x, pro niž byl kód optimalizován. Filtry však pracují s pamětí velmi extenzivně a dochází k \emph{opakovanému čtení} ze stejného umístění (jeden voxel se promítne do výsledku mnoha okolních), na což je globální paměť zcela nevhodná. Díky malému rozsahu dat zpracovávaných jedním blokem tedy používáme ke čtení výhradně texturovou cache (1D, namapovanou na celý obraz), případně kombinovanou s kopírováním celého bloku dat do sdílené paměti.

        Zápis je bezproblémový, protože se jedná vždy jen o jednu hodnotu za vlákno (nebo skupinu vláken) a tudíž i když se přístupy vláken nepodaří sdružit optimálně, jde pouze o jediný zápis.

        \subsubsection{Konstantní paměť}

        Konstantní paměť používáme k uložení všech dlouhodobě neměnných dat -- všech předpočítaných geometrických veličin a část dat masek. U masek neukládáme {\tt wList}, jelikož jeho velikost je proměnná na velké škále a definováním vysoké horní hranice bychom rychle konstatní paměť vyčerpali. {\tt wList} je ovšem na začátku každého kernelu manuálně zkopírován do sdílené paměti pomocí makra {\tt SE\_TO\_SHARED}.

        \subsubsection{Optimalizace registrů}

        Vzhledem k tomu, že algoritmy pro většinu filtrů jednoduché, výsledné kernely jsou poměrně malé a na jeden SM se nám vejde velký počet bloků (nebo alespoň vláken), což je ideální, protože se dobře překryjí paměťové latence. Bloky ale bohužel nejsou paměťově tak malé, aby se uplatnilo hardwarové omezení a zvyšování jejich počtu na jednom SM tak bude střídavě narážet na fyzické omezení množsví sdílené paměti (probráno u konkrétních filtrů) a hlavně registrů\footnote{pro představu naše vlákna spotřebují zhruba 11-25 registrů. To pro CC 1.x (8192 registrů/SM) odpovídá 744-327 vláken/SM, což ještě nenaráží na fyzickou hranici 768 vláken/SM (CC 1.0)}.

        Pro snížení počtu použitých registrů můžeme udělat několik věcí:
        \begin{itemize}
          \item Recyklace proměnných -- proměnné, jejichž obsah není aktuálně potřeba lze dočasně použít např. jako inkrementální proměnné v cyklech.
          \item Řízená výměna rychlosti za menší počet využitých registrů změnou algoritmu (v optimalizaci kompilátoru by ale měla být pořád zapnutá optimalizace na rychlost).
          \item Uložení méně potřebných dat do sdílené paměti, případně uložení konstantních maker do konstantní paměti -- zde je nutné experimentovat, \emph{celková} rychlost může i při větší hustotě bloků/SM klesnout.
          \item Opatrné používání C příkazu \Vr"\bl{goto}" při vyskakování z mnoha cyklů naráz nám také může ušetřit jednu kontrolní proměnnou.
        \end{itemize}

        Pro další experimentování nám pomůže sledování obsazenosti SM pomocí CUDA Occupancy Calculator\footnote{přehledné excelové makro dodávané s CUDA SKD}, kde můžeme zjistit, na jakou z výše zmíněných hranic právě narážíme. Zde můžeme například zjistit, že na SM se již další blok nevejde, a tak naopak navýšit počet registrů a zvýšit tak rychlost. Mírnou nevýhodou je, že tyto jemné optimalizace už jsou zcela závislé na konkrétní CC, protože každá má jiné množství registrů a sdílené paměti.

        Další nepříjemností, co se týče uvolňování registrů jsou systémové proměnné typu \Vr"\cy{thread-}"\linebreak\Vr"\cy{Idx}" a \Vr"\cy{blockIdx}", které zřejmě\footnote{při porovnávání počtu proměnný (a maker, která kompilátor pravděpodobně kvůli rychlosti uloží do registrů) a počtu obsazených registrů opakovaně přebývalo několik registrů, které byly zřejmě použity právě pro ID vlákna} v registrech zůstávají, byť je u jednodušších kernelů potřebujeme jen na začátku kernelu. Navíc se ani nemohou účastnit recyklace, protože do nich není možné zapisovat. Z tohoto důvodu je při extrémní minimalizaci počtu registrů praktické volit dimenzionalitu gridu a bloku co nejmenší, aby se jejich počet omezil (s tím je ale nutné počítat od začátku návrhu kódu).

    \subsection{Optimalizace běhu jádra}

    Následující sekce obsahuje několik dalších optimalizací, které se při experimentování s kernely osvědčily. \note{nejsou už tolik univerzální?}

        \subsubsection{Synchronizace}

        Několik poznámek k instrukci \Vr"\cy{__syncthreads()}":
        \begin{itemize}
          \item Rozhodně s nimi šetřit, neboť nejsou časově nejlevnější -- sice zaberou jen 8 taktů \cite{CUDA programming g.}, ale jak jsme zmiňovali, přeruší cyklus překrývání latencí, což může způsobit značné zpomalení.
          \item Pokud se program větví, každá cesta v kódu musí obsahovat \emph{stejný počet} \Vr"\cy{__syncthre-}"\linebreak\Vr"\cy{ads()}", protože GPU k prohlášení bloku za synchronizovaný stačí, aby každé vlákno dorazilo k \emph{nějaké} synchronizační instrukci, tzn. nemusí to být u všech vláken ta samá. Porušení tohoto vede ve většině případů k pádu programu.
          \item Speciálním případem předchozího je, že jedno vlákno zavolá \Vr"\bl{return}" a ostatní na něj při synchronizaci marně čekají (až do resetu GPU).
        \end{itemize}
        V našich memory-bound kernelech se tedy budeme použití \Vr"\cy{__syncthreads()}" vyhýbat, pokud to půjde.

        \subsubsection{Větvení programu}\label{vetvení}

        Z předchozího je jasné, že větvení je nejlépe omezit pouze na hranice warpů, pokud už nejde zcela vynechat. Je logické, že nejvíce musíme větvení optimalizovat v nejvytíženějších částech kódu. Příklad:
        \begin{itemize}
        \item Podmínku na začátku kódu ověřující, zda má vlákno ještě vůbec nějaká data ke zpracování, většinou vynechat nelze, ale také nijak nezhoršuje výkon -- dotkne se pouze několika posledních waprů v posledním bloku. Toto se týká obecně podmínek rozumně závisejících pouze na ID vlákna (\Vr"\cy{threadIdx}",\Vr"\cy{blockIdx}").
        \item Cyklus obsahující několik (krátkých, nejlépe jednoinstrukčních) podmínek závislých na vnitřních proměnných naproti tomu bude generovat poloprázdné warpy po dobu běhu celého kernelu a zde je na místě pokusit se větve odstranit.
        \end{itemize}
        Jeden z případů druhé skupiny, kde se dá větvení zcela odstranit je následující:
        \begin{Verbatim}[commandchars = \\\{\}]
\bl{if}(x<y) z++;
        \end{Verbatim}
        lze ekvivalentně přepsat na
        \begin{Verbatim}[commandchars = \\\{\}]
z += (x<y);
        \end{Verbatim}
        kde se spoléháme na normu\footnote{nevíme, jeslti je \emph{zaručeno}, že ji kompilátor dodržuje, nebo zda je to machine-dependent záležitost -- nutno vyzkoušet}, že podmínka má hodnotu 1, pokud platí a 0, pokud ne. Pokud se takovéto podmínky nalézají ve vytížené části kódu, může být urychlení velmi znatelné. Zdá se totiž, že syntaxi s \Vr"\bl{if}" kompilátor nezvládne zoptimalizovat a vytvoří dvě větve -- výsledek podmínky použije pro podmíněný skok -- zatímco v druhém případě mu explicitně říkáme, jak má výsledek podmínky použít. Ze stejných důvodů je výhodné používat ternární operátor \Vr"?" kdekoliv je to možné.

        Narozdíl od CPU není vhodné tvořit z několika složitých po sobě jdoucích podmínek stromy obsahující jednodušší podmínky -- vlákno sice musí prozkoumat v průměru méně podmínek, aby stromem prošlo, vznikne tím ovšem hodně \emph{divergentních vláken} (po cestě, kterou se vydaly nejdou skoro žádná další vlákna z warpu) a výpočty budou serializovány, protože se vytvoří mnoho řídce obsazených warpů. Ponecháme-li složité podmínky v kódu volně za sebou, bude sice vyhodnocení samotné podmínky trvat déle, ale bude vykonáno všemi vlákny naráz, protože ty se mezi podmínkami synchronizují a tudíž bude kód ve výsledku ryhlejší.
        \begin{Verbatim}[commandchars = \\\{\}]
// rychlejší varianta
\bl{if}(...) && (...)\{
...     // dvě (silné) větve
\}
    // zde se vlákna synchronizují
\bl{if}(...) && (...)\{
...     // dvě (silné) větve
\}

// pomalejší varianta
\bl{if}(...)\{
    \bl{if}(...)\{
    ...      // tři větve
    \}\bl{else if}(...)\{
    ...
    \}
\}
        \end{Verbatim}

\section{Implementace filtrů na GPU}

    Zde se konečně dostáváme ke popisu kernelů provádějících filtraci. Nyní v bodech probereme, co máme před začátkem vlastní filtrace k dispozici a co musíme při psaní kódu zohlednit:
    \begin{itemize}
      \item Každý vstupní voxel chceme zpracovávat jedním vláknem, nebo skupinou vláken.
      \item 3D data a {\tt wList} z masky jsou na GPU v globální paměti uložena ve stejném formátu jako na CPU a je k nim možno stejným způsobem přistupovat.
      \item Kernelu je předána reference na texturu, která byla svázána se zpracovávanými daty.
      \item V konstantní paměti GPU máme připravenou veškerou potřebnou geometrii a konstatní data masek.
      \item Po inicializační části vlákna máme k dispozici {\tt wList} použité masky ve sdílené paměti a v proměnné {\tt thread} index zpracovávaného voxelu vypočtený z ID vlákna.
    \end{itemize}

    Konkrétní kód inicializační části kernelu zde popisovat nebudeme, neboť její průběh není přiliš podstatný a navíc je kód díky optimalizacím (rychlost, recyklace proměnných) lehce nepřehledný. Nyní již k jednotlivým filtrům.

    \subsection{Morfologické filtry}

    Popis těchto filtrů je možné najít v sekci~\ref{sec morph}. Jejich implementace je opět triviální: každé vlákno počítá hodnotu jednoho výstupního voxelu. Vlákno v cyklu načítá z textury podle {\tt wList} hodnoty příslušných okolních voxelů a z nich vybere minimum (eroze), maximum (dilatace), nebo oboje (detekce hran). Vzhledem k velkému množství přístupu do paměti ku počtu jiných operací (jedna, nebo dvě podmínky) se u těchto filtrů nedaří zcela překrýt paměťové latence. To má za následek, že v případě detekce hran je přidání druhé podmínky (jedna hledá maximum, druhá minimum) \bq zdarma\eq, neboť jinak by SM stejně nic nedělal a čekal na zpracování paměťových přístupů. Množství přístupů do paměti je ovšem u detekce hran \emph{stejné}, jako u dilatace a eroze.

    Díky tomu ve výsledcích uvidíme, že detekce hran má takřka dvojnásobné relativní zrychlení vůči CPU (kde se podmínka navíc samozřejmě projeví), než dilatace, nebo eroze. U nich tedy má smysl uvažovat o snížení paměťových latencí. To by se dalo udělat například zkopírováním bloku zpracovávaných dat do sdílené paměti -- pro maximální efektivitu to však vyžaduje poměrně komplikovanou přestavbu kódu, včetně přepočítání masek a jiné režie, takže jsme od dalších optimalizací prozatím upustili.

    \subsection{Medián a BES}

    Popis těchto filtrů je možné najít v sekci~\ref{statisticky motivované}. Součástí těchto filtrů je částečné třídění pole. V současnosti existuje spousta literatury popisující paralelizaci známých třídících algoritmů, jako quicksort, mergesort, bitonic sort apod., které se však zaměřují na třídění obřích polí (10$^6$ až 10$^9$ prvků) a zřídka bývají efektivní pro malá pole. V našem případě však potřebujeme paralelně setřídit \emph{mnoho malých} polí, a to nejlépe při malé spotřebě sdílené paměti, což hardwaru umožní přidělit více bloků na jedno SM a lépe tak překrýt latence.

    Pro malé velikosti polí (destíky prvků) jsou obecně výhodnější \bq hloupější\eq ~\OOO($n^2$) algoritmy, protože mají oproti složitějším algoritmům velmi malou režii. Při použití na GPU na ně navíc klademe požadavek, aby jejich průběh byl pokud možno nezávislý na vstupních datech (co nejméně větví, nebo alespoň stejně dlouhé) -- to zamezí přílišnému rozdrobení warpů, jejich serializaci a následnému zpomalení. Jako vítěz se při těchto požadavcích jeví algoritmus zapomětlivého třídění \cite{Forgetful}.

        \subsubsection{Zapomětlivé třídění}

        Popis algoritmu provedeme pro pole liché délky (medián je jediný prvek -- viz~\ref{def median}):
        buď \textbf{medián} $\lfloor \frac{c+1}{2} \rfloor$-tým prvkem v setříděném poli (označme tento prvek jako \kk-tý), princip jeho nalezení je následující:
        \begin{enumerate}
          \item Z nesetříděného pole vyberme libovolně \kk$+1$ prvků (nejjednodušeji od začátku pole).
          \item Mezi vybranými prvky najděme maximum a minimum a ty zahoďme -- hledání můžeme provést např. jako první krok insertion-sortu.
          \item Do výběru přidejme jeden prvek z nesetříděného pole, který jsme ještě nevybrali (lhostejno kam, třeba na konec).
          \item Opakujme kroky 2. a 3. dokud nezbude pouze jeden prvek\footnote{dva pro případ sudé velikosti pole}, ten je hledaným mediánem.
        \end{enumerate}
    
        Úprava algoritmu pro sudou velikost pole je popsána v poznámce pod čarou a je jediná, pokud \kk~volíme tak, že medián se bude konstruovat z \kk-tého a (\kk$-1$)-tého prvku.
         
        Při hledání se tedy pole neustále zmenšuje, až zbyde pouze medián. Navíc nepotřebujeme současně celé pole, ale pouze něco přes polovinu, což kvůli latencím umožní zvýšit rychlost až dvakrát (!) oproti algoritmům využívajícím celé pole. Algoritmus je jistě \OOO($n^2$) -- počet hledání maxim a minim je roven součtu aritmetické řady. Také je jasné, že algoritmus se větví pouze při podmínkách nutných k nalezení maxima a minima -- zde se osvědčil operátor \Vr"?".

        Nyní dokažme, že algoritmus nikdy nezahodí medián -- budiž počet prvků v poli lichý, pro sudý to platí obdobně:

        V prvním kroku zaručujeme, že maximum i minimum budou jistě nad a pod hodnotou mediánu (ať už on sám je vybrán, nebo ne), protože vybraných prvků je \bq moc\eq ~-- nejhorší možné výběry ukazuje obrázek~4.1.
        \begin{figure}[h]
        \begin{center}
          \includegraphics[width = 0.75\textwidth]{src/4Implementace/forgetful.pdf}
          \caption{Nejhorší výběry při setříděném poli, modrá značí medián}
          \end{center}
        \end{figure}\label{obr forgetful}
        Vyhozením minima a maxima tedy o medián nepřijdeme. Uvidíme, že toto platí v každém dalším kroku:
        \begin{enumerate}
          \item Medián je mezi vybranými prvky:
            \begin{enumerate}
                \item Po zahození maxima a minima je medián stále uvnitř výběru (není minimem ani maximem z toho, co zbylo): při dalším kroku nebude zahozen.
                \item Po zahození maxima a minima je medián maximum (pokud je minimem, platí tvrzení analogicky) ze zbytku: hodnota přidaná z ještě nesetříděných bude nutně větší než medián, neboť \emph{všechny} menší již ve výběru jsou (viz obrázek~4.1). Tudíž při dalším zahazování medián opět zůstane ve výběru.
            \end{enumerate}
          \item Medián není mezi vybranými prvky, ale je mezi ně přidán po zahození maxima a minima: ve výběru je příliš mnoho prvků, a tak všechny nemohou být zároveň menší, nebo zároveň větší než přidávaný medián a ten tudíž nebude v dalším kroku maximem.
        \end{enumerate}

        Zapomětlivé třídění lze jednoduše modifikovat pro \textbf{BES} filtr: Na začátku stačí vzít tolik hodnot, abychom zajistili, že i první a třetí kvartil budou \emph{uvnitř} výběru (viz obrázek~4.2), poté provádíme
        \begin{figure}[h]
        \begin{center}
          \includegraphics[width = 0.75\textwidth]{src/4Implementace/forgetfulBES.pdf}
          \caption{Nejhorší výběry při setříděném poli, BES, modrá značí kvartily}
          \end{center}
        \end{figure}\label{obr forget BES}
        klasické zapomětlivé třídění. Po přidání poslední nesetříděné hodnoty je jasné, že nalezené maximum a minimum odpovídá třetímu a prvnímu kvartilu. Po jejich zapamatování dále již maxima a minima pouze vyhazujeme (nemáme co přidávat), dokud nezbyde medián. Argumentace správnosti je obdobná, jako v předchozím případě.

        Úspora paměti je v tomto případě menší -- limitně čtvrtina -- i tak má však algoritmus velmi slušné výsledky.

    \subsection{Hodges-Lehmannův medián a WBES}

     Popis těchto filtrů je možné najít v sekci~\ref{statisticky motivované}. Protože jsou postavené na Walshově seznamu (který na začátku paralelně vytvoříme, viz příloha~\ref{struktura kódu}), je tříděné pole již poměrně velké (stovky prvků). Kvůli tomu nám při přístupu jedno pole/jedno vlákno rapidně klesne vytížení SM, protože se do sdílené paměti vejde málo polí, a navíc se projeví pomalost \OOO($n^2$) algoritmů. Pro přístup pole/blok\footnote{což je také minimální hranice pro použití sofistikovanějších algoritmů} je zase pole příliš malé -- idelální je mnohonásobek velikosti bloku, zde máme v praxi něco kolem dvojnásobku.

    Jako kompromis jsme zvolili přístup pole/warp (nebo obecně skupina vláken) a použití velmi naivního algoritmu: pro každý prvek pole jednoduše spočítáme, kolik hodnot je ostře menší a ostře větších než tento prvek a na jejich základě určíme, zda je daný prvek mediánem (nebo libovolným jiným \kk-tým prvkem). Algoritmus má sice zcela jasně složitost \OOO($n^2$), ale dá se pro GPU celkem pěkně optimalizovat; popišme nyní jeho implementaci (popis jednoho warpu -- 32 vláken):
    \begin{enumerate}
      \item Ze vstupních dat vytvoříme ve sdílené paměti Walshův seznam (WS).
      \item Každé vlákno si zapamatuje jeden z 32 prvků ze začátku WS.
      \item V cyklu porovnává každé vlákno zapamatovaný prvek postupně se všemi prvky WS, počítá ostře menší a ostře větší prvky. Vlákna přitom postupují synchronně a v každém kroku je porovnávaný prvek pro všechna vlákna stejný.
      \item Každé vlákno ověří, zda nenalezlo medián (případně jiný potřebný \kk-tý prvek), pokud ano, zapíše ho do paměti.
      \item Pokud byly nalezeny všechny hledané prvky, warp už jen počká na konec celého bloku.
      \item V opačném případě načte warp dalších 32 prvků z WS a pokračuje na krok 3.
    \end{enumerate}
    \begin{figure}[h]
    \begin{center}
      \includegraphics[width = 0.75\textwidth]{src/4Implementace/scansort.pdf}
      \caption{Schéma algoritmu pro Hodges-Lehmannův medián a WBES}
    \end{center}
    \end{figure}

    Z definice mediánu (\ref{def median}) je zřejmé, že daný prvek je medián, právě když platí následující dvě nerovnosti ($c$ je délka pole, $p_{(<)}$ označuje počet prvků ostře menších než testovaný prvek, $p_{(>)}$ počet ostře větších):
    \begin{align}
      p_{(<)} < \Big\lfloor\frac{c+1}{2}\Big\rfloor \notag \\
      p_{(>)} < \Big\lceil\frac{c+1}{2}\Big\rceil \notag
    \end{align}
    Obdobné nerovnosti lze stanovi i pro jakýkoliv jiný \kk-tý prvek.

    Výhodou této imlementace je, že v časově nejnáročnější části (krok 3.) se algoritmus díky použití optimalizace~\ref{vetvení} vůbec nevětví a navíc minimálně přistupuje do paměti (načtení pouze jediného prvku pro celý warp). Navíc díky zapisování nalezených hodnot může algoritmus skončit ještě dříve, než porovná \emph{všechny} prvky, což mírně sníží konstantu u časové náročnosti. Bohužel, algoritmus taktéž potřebuje hodně synchronizačních bodů a celkově se pohybuje na hraně efektivity kvůli špatným rozměrům vstupu -- to by se ale dalo částečně vyřešit použitím lepší GPU.

    Je zřejmé, že popsaný algorimus lze použít jak pro H-L mediá, tak pro WBES -- vše záleží pouze na volbě podmínek v kroku 4.

    \chapter{Výsledky}
        % volba masky 
        % Testovací data
        % Testovací sestava
        % Urychlení vůči CPU

    \addcontentsline{toc}{chapter}{Závěr}
    \chapter*{Závěr}

    \addcontentsline{toc}{chapter}{Literatura}

    % -*-coding: utf-8 -*-
\begin{thebibliography}{9}
    \bibitem{MajerovaPhD}
        bla bla

    \bibitem{Bělíček}
        fuzzy edge detectors
        
    \bibitem{Charypar}
        fuzzy watershed

\end{thebibliography} 

    % CITACE:

\end{document} 

% def k-tá statistika
% dl, že operace LA splňují residuovaný svaz.