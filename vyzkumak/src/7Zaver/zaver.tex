% -*-coding: utf-8 -*-

Podařilo se nám vytvořit vysoce funkční model pro paralelizaci optimalizačních algoritmů. V popsaném formalismu založeném na struktuře evolučních algoritmů jsme schopni popsat, rozložit a posléze v modelu implementovat celou škálu optimalizačních algoritmů. Díky důrazu na stavebnicový princip můžeme algoritmus dekomponovat na malé části, jejichž implementace a paralelizace je jednoduchá, nebo je možné při ní použít osvědčené algoritmy. Vzniklé komponenty jsou elegantně malé opakovaně použitelné i v jiných OA. 

Při testování zrychlení za použití GPU na třech algoritmech (RS, GO, SA) a dvou účelových funkcích (De Jong č.1, sudoku) jsme dosáhli jednak očekávaných výsledků zrychlení na úrovni 100, tak i výsledků překvapivých, kdy maximální zrychlení na problému sudoku dosáhlo hodnoty 566krát (tedy nad očekávanou maximální hranicí). Tento úspěch naznačuje, že by model mohl být dobře použitelný i na řešení velkých úloh pocházejících z praxe za předpokladu, že budou dobře paralelizovatelné. 

Navržený model je snadno rozšiřitelný, životaschopný a může sloužit jako základní kámen pro statistické testování kvality různých OA, které vyžaduje extenzivní výpočty.  